\documentclass[13pt, t]{beamer}

% Presento style file
\usepackage{config/presento}

% custom command and packages
% custom packages
\usepackage{textpos}
\setlength{\TPHorizModule}{1cm}
\setlength{\TPVertModule}{1cm}

\newcommand\crule[1][black]{\textcolor{#1}{\rule{2cm}{2cm}}}



% Information
\title{\Large Gender attribution of borrowings and native lexemes in Zilo (Andi)}
\author[shortname]{Samira Verhees and George Moroz \bigskip}
\institute[shortinst]{Linguistic Convergence Laboratory, NRU HSE, Moscow, Russia}
\date{\begin{center} 11 May 2018 г. \bigskip \\ {\color{colorblue} Anatolia-the Caucasus-Iran: Ethnic and Linguistic Contacts \\ Institute of Oriental Studies in Russian-Armenian State University} \end{center}}

\begin{document}

% Title page
\begin{frame}[plain]
\maketitle
\end{frame}

\framepic{images/01-map-all}

\begin{frame}{Andi and their neighbors}
\Large
\begin{itemize}
\item Nakh
\begin{itemize}
\item \Large Chechen
\end{itemize}
\item Avaro-Andic-Tsezic
\begin{itemize}
\item \Large Avar
\item \Large Andi
\begin{itemize}
\item \Large Botlikh
\item \Large Karata
\item \Large Andi
\end{itemize}
\end{itemize}
\end{itemize} \pause
All Andi speakers are multilingual:
\begin{itemize}
\item Avar --- old lingua franca of a region
\item Russian --- new lingua franca of a whole Daghestan (2 centuries of active Russification)
\end{itemize}
\end{frame}

\framecard[colorblue]{{\color{colorwhite} \huge Gender systems}}

% Введение
\begin{frame}{Gender systemes}
\Large 
\begin{itemize}
\item The impact the systems have most notably on verbal agreement, but also on some adjective, numeral, adverbial, pronominal, and postpositional agreement. 
\item Some POS agree with absolutive argument of the clause. Some  POS agree with its head.
\item In most cases gender is a covert category, but there are some exception, such as:
\begin{itemize}
\item \textit{w-otsːi} (\textsc{m}-sibling) 
\item \textit{j-otsːi} (\textsc{f}-sibling)
\item \textit{b-otsːi} (\textsc{an}-sibling) --- from the ``Three Little Pigs'' fairytale
\end{itemize}
\end{itemize}
\end{frame}

\begin{frame}{Andi gender inventory}
\small
\begin{tabular}{|l|l|c|c|c|c|c|c|c|c|}
\hline
\multicolumn{3}{|c|}{}  & \textsc{m} & \textsc{f} & \textsc{an} & \textsc{¬an} 1 & \textsc{¬an} 2 & \textsc{¬an} 3 & \textsc{¬h} \\ \hline
\multirow{2}{*}{\cite{alekseev99}} & \multirow{ 2}{*}{Andi} & \textsc{sg} & \multirow{2}{*}{\textit{w}} & \multirow{2}{*}{\textit{j}} & \textit{b} & \multirow{2}{*}{\textit{b}} & \multirow{2}{*}{\textit{r}} & \multicolumn{2}{|c|}{\multirow{2}{*}{\textit{-}}}  \\ \cline{3-3} \cline{6-6}
 &  & \textsc{pl} & \textit{} & \textit{} & \textit{j} & \textit{} & \textit{} & \multicolumn{2}{|c|}{} \\ \hline
\multirow{2}{*}{\cite{salimov10}} & \multirow{2}{*}{Gagatli} & \textsc{sg} & \multirow{2}{*}{\textit{w}} & \multirow{2}{*}{\textit{j}} & \textit{b} & \multirow{2}{*}{\textit{b}} & \multirow{2}{*}{\textit{r}} & \multicolumn{2}{|c|}{\multirow{2}{*}{\textit{-}}} \\ \cline{3-3} \cline{6-6}
 &  & \textsc{pl} & \textit{} & \textit{} & \textit{j} & \textit{} & \textit{} & \multicolumn{2}{|c|}{} \\ \hline
\multirow{2}{*}{\cite{suleymanov57}} & \multirow{2}{*}{Rikwani} & \textsc{sg} & \multirow{2}{*}{\textit{w}} & \textit{j} & \textit{b} & \multirow{2}{*}{\textit{b}} & \multirow{2}{*}{\textit{r}} & \textit{b} & \multirow{2}{*}{\textit{-}} \\ \cline{3-3} \cline{6-6} \cline{9-9}
 & & \textsc{pl} & \textit{} & \textit{} & \textit{j} & \textit{} & \textit{} & \textit{r} &  \\ \hline
\multirow{2}{*}{fieldwork} & \multirow{2}{*}{Zilo} & \textsc{sg} & \multirow{2}{*}{\textit{w}} & \multirow{2}{*}{\textit{j}} & \textit{b} & \multirow{2}{*}{\textit{b}} & \multirow{2}{*}{\textit{r}} & \multicolumn{2}{|c|}{\multirow{2}{*}{\textit{-}}}  \\ \cline{3-3} \cline{6-6}
 & & \textsc{pl} & \textit{} & \textit{} & \textit{j}& \textit{} & \textit{} & \multicolumn{2}{|c|}{}  \\ \hline
\multirow{2}{*}{\cite{suleymanov57}} & \multirow{2}{*}{Muni} & \textsc{sg} & \multirow{2}{*}{\textit{w}} & \multirow{2}{*}{\textit{j}} & \multicolumn{4}{|c|}{\multirow{2}{*}{\textit{-}}}  & \multirow{2}{*}{\textit{b}} \\ \cline{3-3}
 & & \textsc{pl} & \textit{} & \textit{} & \multicolumn{4}{|c|}{} &  \\ \hline
\end{tabular}
\normalsize
\begin{itemize}
\item \textsc{m} vs. \textsc{f} vs. \textsc{¬an} vs. \textsc{¬an}s 
\item \textsc{sg} = \textsc{pl} (except \textsc{an} and \textsc{¬an 3})
\item Rikwani's \textsc{¬an} 3 gender  contains only several words: \textit{hotʃ'ortʃ'in} `scorpion', \textit{odoruk'a} `butterfly', and some other 5 words

\item In ``lower-group'' (Muni and Kvanxidatli) there are only three genders  (\textsc{m} vs. \textsc{f} vs. \textsc{¬h})
 
% СФ: не знаю куда хочешь впихнуть эту информацию, просто так она не влезет, слишком много.  
% СФ: \item По  система верхних (с четыремя показателями) отражает исходную ситуацию аваро-андийско-цезской группы. 

% ГМ: Самирча, родная, я не хочу это вставлять, потому что это мутное утверждение, я не знаю, как можно было бы красиво его доказать. Давай оставим это в нашей памяти, но сейчас я отвечаю на другой вопрос.

\end{itemize}
\end{frame}



\begin{frame}{Neighbour languages gender inventories}
\begin{tabular}{|l|l|c|c|c|c|c|c|c|}
\hline
\multicolumn{ 3}{|l|}{}& \textsc{m} & \textsc{f} & \multicolumn{ 4}{c|}{\textsc{¬h}} \\ \hline
\multirow{ 2}{*}{\cite[41]{alekseev97}} & \multirow{ 2}{*}{Avar} & \textsc{sg} & \textit{w} & \textit{j} & \multicolumn{ 4}{c|}{\textit{b}} \\ \cline{ 3- 9}
\multicolumn{ 1}{|l|}{} & \multicolumn{ 1}{l|}{} & \textsc{pl} & \multicolumn{ 6}{c|}{\textit{r/l}} \\ \hline
\multirow{ 2}{*}{\cite[41]{magomedbekova71}} & \multirow{ 2}{*}{Karata} & \textsc{sg} & \textit{w} & \textit{j} & \multicolumn{ 4}{c|}{\textit{b}} \\ \cline{ 3- 9}
\multicolumn{ 1}{|l|}{} & \multicolumn{ 1}{l|}{} & \textsc{pl} & \multicolumn{ 2}{c|}{\textit{b}} & \multicolumn{ 4}{c|}{\textit{r}} \\ \hline
\multirow{ 2}{*}{\cite[253--254]{gudava62}} & \multirow{ 2}{*}{Botlikh} & \textsc{sg} & \textit{w} & \textit{j} & \multicolumn{ 4}{c|}{\textit{b/m}} \\ \cline{ 3- 9}
\multicolumn{ 1}{|l|}{} & \multicolumn{ 1}{l|}{} & \textsc{pl} & \multicolumn{ 2}{c|}{\textit{b}} & \multicolumn{ 4}{c|}{\textit{r/n}} \\ \hline
\multirow{ 2}{*}{\cite[21--22]{nichols94}} & \multirow{ 2}{*}{Chechen} & \textsc{sg} & \textit{w} & \textit{j} & \multirow{ 2}{*}{\textit{j}} & \multirow{ 2}{*}{\textit{d}} & \multirow{ 2}{*}{\textit{b}} & \textit{b} \\ \cline{ 3- 5} \cline{ 9- 9}
\multicolumn{ 1}{|l|}{} & \multicolumn{ 1}{l|}{} & \textsc{pl} & \textit{b} & \textit{d} &  & & & \textit{d} \\ \hline
\end{tabular}
\vfill
\begin{itemize}
\item \textsc{m} vs. \textsc{f} vs. \textsc{¬h}
\item \textsc{sg} ≠ \textsc{pl} (except Chechen)
\end{itemize}
\end{frame}

\framepic{images/02-zilo-classes}{}

\framecard[colorblue]{{\color{colorwhite} \huge Gender attribution experiment}}

\begin{frame}{Experiment: set up}
\alert{How speakers decide about new words?}
\begin{itemize}
\item collect 25 native and 25 loan words of \textsc{an} 1 gender
\item collect 25 native and 25 loan words of \textsc{an} 2 gender
\item add some words of Rikwani's \textsc{an} 3 gender
\item pseudo random stimuli order
\item 16 Zilo native speakers
\item context:  \textit{di-b} / \textit{di-r} X ``my  X''
\end{itemize}
\end{frame}

\begin{frame}{Experiment: words}
\begin{multicols}{2}
\begin{itemize}
\item[22]    kots'i broom
\item[48]    awʔara cap
\item[4] ʁats'a grasshopper
\item[13]      ts'a fire
\item[68]      nitʃo scythe
\item[57]    rok'o heart
\item[92]        ɬen water
\item[29]     hark'u eye
\item[46]      tɬ'ir bridge
\item[25]      q'en wall
\item[85]     reʃa tree
\item[28]    kʷeɬir sleeve
\item[79]    hunts'ːi honey
% loans
\item[98]       raketa rocket
\item[106]       kolxoz collective farm
\item[39] 	kalendarʲ calendar
\item[83]         sumka bag
\item[65]       maʃina car
\item[58]    pasport passport
\item[71]       simkarta  sim-card
\item[16]      kirpitʃ brick
\item[63]         tarif tariff
\item[105]    rozʲetka  power outlet
\item[45]        patʃka packet
\item[64]      esemeska sms
\item[75]   plastɨrʲ plaster
\end{itemize}
\end{multicols}
\end{frame}

\begin{frame}{Inter-rater reliability}
Заданные вопросы можно было бы переиначить в рамках статистической задачи измерения согласия. Мы будем использовать метода перечисленные в работе \citep{hallgren12}.

\begin{block}{Процент согласия}
высчитывается как количество случае абсолютного согласия:
$\frac{79}{106} = 0.745283$
\end{block}
\begin{block}{Fleiss' κ}
Fleiss' κ for 16 Raters = 0.849 , z = 95.7, p-value = 0
\begin{itemize}
\item    Poor agreement — Less than 0.20
\item    Fair agreement — 0.20 to 0.40
\item    Moderate agreement — 0.40 to 0.60
\item    Good agreement — 0.60 to 0.80
\item    Very good agreement — 0.80 to 1.00
\end{itemize}
\end{block}
\end{frame}

\begin{frame}{Слова с вариативностью}
27 вариативных слов; 16 с более чем одним респондентами: \bigskip\\

\small
\begin{tabular}{|l|l|l|l|l|l|}
\hline
id & зиловский & русский & источник & b & r \\ \hline
45 & patʃka & пачка & loan & 3 & 13 \\ \hline
64 & antena & антенна & loan & 3 & 13 \\ \hline
101 & paket & пакет & loan & 3 & 13 \\ \hline
17 & χur & поле, огород & native & 4 & 12 \\ \hline
44 & provod & провод & loan & 4 & 12 \\ \hline
70 & ɡol & гол & loan & 5 & 11 \\ \hline
21 & simkarta & сим-карта & loan & 6 & 10 \\ \hline
24 & zatʃetka & зачетка & loan & 6 & 10 \\ \hline
2 & vaɡon & вагон & loan & 9 & 7 \\ \hline
20 & odoruk'a & бабочка & native & 9 & 7 \\ \hline
39 & kalendarʲ & календарь & loan & 9 & 7 \\ \hline
75 & plastɨrʲ & пластырь & loan & 10 & 6 \\ \hline
41 & sputnʲik & спутник & loan & 12 & 4 \\ \hline
98 & raketa & ракета & loan & 12 & 4 \\ \hline
71 & kometa & комета & loan & 14 & 2 \\ \hline
72 & lift & лифт & loan & 14 & 2 \\ \hline
\end{tabular}
\end{frame}

\begin{frame}{Предварительные выводы и вопросы}
\begin{itemize}
\item Вариативность в области классной категоризации есть.
\item Вариативность сосредоточена в области заимствованных слов.
\item Вариативность симметрична. Нет крена в сторону какого-то одного класса.
\item Интересно сравнить с результатами экспериментов с искусственными словами (ср. польский родительный в работе\\ \ \hfill\citep{dabrowska05})
\item \textit{odoruk'a} `бабочка'! Что с другими словами 6-ого рикванинского класса?
\end{itemize}
\end{frame}

\framecard[colorblue]{{\color{colorwhite} \huge Спасибо за внимание! \bigskip\\
\Large Пишите письма\\
agricolamz@gmail.com\\
jh.verhees@gmail.com}}

%\begin{frame}[allowframebreaks]{Список литературы}
\begin{frame}{Список литературы}
\footnotesize
\bibliographystyle{chicago}
\bibliography{bibliography}
\end{frame}

\end{document}